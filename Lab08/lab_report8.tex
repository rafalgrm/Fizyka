%%%%%%%%%%%%%%%%%%%%%%%%%%%%%%%%%%%%%%%%%
% University/School Laboratory Report
% LaTeX Template
% Version 3.1 (25/3/14)
%
% This template has been downloaded from:
% http://www.LaTeXTemplates.com
%
% Original author:
% Linux and Unix Users Group at Virginia Tech Wiki 
% (https://vtluug.org/wiki/Example_LaTeX_chem_lab_report)
%
% License:
% CC BY-NC-SA 3.0 (http://creativecommons.org/licenses/by-nc-sa/3.0/)
%
%%%%%%%%%%%%%%%%%%%%%%%%%%%%%%%%%%%%%%%%%

%----------------------------------------------------------------------------------------
%	PACKAGES AND DOCUMENT CONFIGURATIONS
%----------------------------------------------------------------------------------------

\documentclass{article}

\usepackage{polski}
\usepackage[utf8]{inputenc}
\usepackage{booktabs}
\usepackage{multirow}
\usepackage{caption}

\usepackage[version=3]{mhchem} % Package for chemical equation typesetting
\usepackage{siunitx} % Provides the \SI{}{} and \si{} command for typesetting SI units
\usepackage{graphicx} % Required for the inclusion of images
\usepackage{natbib} % Required to change bibliography style to APA
\usepackage{amsmath} % Required for some math elements 

\setlength\parindent{0pt} % Removes all indentation from paragraphs

\renewcommand{\labelenumi}{\alph{enumi}.} % Make numbering in the enumerate environment by letter rather than number (e.g. section 6)
\usepackage{color}

%\usepackage{times} % Uncomment to use the Times New Roman font

%----------------------------------------------------------------------------------------
%	DOCUMENT INFORMATION
%----------------------------------------------------------------------------------------

\title{Ćwiczenie nr 82: Efekt fotoelektryczny} % Title

\author{Rafał \textsc{Grabiański} i Zbigniew \textsc{Królikowski}} % Author name

\date{\today} % Date for the report

\addtolength{\oddsidemargin}{-.875in}
\addtolength{\evensidemargin}{-.875in}
\addtolength{\textwidth}{1.75in}
\addtolength{\topmargin}{-.875in}
\addtolength{\textheight}{1.75in}

\begin{document}

% Please add the following required packages to your document preamble:
% \usepackage{booktabs}
\begin{table}[h]
\begin{tabular}{@{}llllll@{}}
\toprule
\begin{tabular}[c]{@{}l@{}}Wydział:\\ \\ WIEiT\end{tabular}                                    & \multicolumn{2}{l}{\begin{tabular}[c]{@{}l@{}}Imię i nazwisko:\\ Rafał Grabiański\\ Zbigniew Królikowski\end{tabular}}                & \begin{tabular}[c]{@{}l@{}}Rok:\\ \\ II\end{tabular}            & \begin{tabular}[c]{@{}l@{}}Grupa:\\ \\ 7\end{tabular}              & \begin{tabular}[c]{@{}l@{}}Zespół:\\ \\ 7\end{tabular} \\ \midrule
\multicolumn{1}{|c|}{\begin{tabular}[c]{@{}c@{}}PRACOWNIA\\ FIZYCZNA\\ WFiIS AGH\end{tabular}} & \multicolumn{4}{l|}{Temat: Ładunek właściwy elektronu $\frac{e}{m}$}                                                                                                                                                                                                                                                  & \multicolumn{1}{l|}{Nr ćwiczenia: 45}                     \\ \midrule
\begin{tabular}[c]{@{}l@{}}Data wykonania:\\ \\ \\ 9.12.2014\end{tabular}                     & \begin{tabular}[c]{@{}l@{}}Data oddania:\\ \\ \\ 16.12.2014\end{tabular} & \begin{tabular}[c]{@{}l@{}}Zwrot do poprawy:\\ \\ \\ .\end{tabular} & \begin{tabular}[c]{@{}l@{}}Data oddania:\\ \\ \\ .\end{tabular} & \begin{tabular}[c]{@{}l@{}}Data zaliczenia:\\ \\ \\ .\end{tabular} & OCENA:                                                  \\ \bottomrule
\end{tabular}
\end{table}

%\maketitle % Insert the title, author and date

% If you wish to include an abstract, uncomment the lines below


%----------------------------------------------------------------------------------------
%	SECTION 1 - CEL ĆWICZENIA
%----------------------------------------------------------------------------------------

\section{Cel ćwiczenia}

Celem ćwiczenia było wyznaczenie ładunku właściwego elektronu $\frac{e}{m}$ za pomocą zbadania ruchu wiązki elektronów w jednorodnym polu magnetycznym wytworzonym przez układ cewek Helmholtza.

% If you have more than one objective, uncomment the below:
%\begin{description}
%\item[First Objective] \hfill \\
%Objective 1 text
%\item[Second Objective] \hfill \\
%Objective 2 text
%\end{description}

 

%\clearpage

%----------------------------------------------------------------------------------------
%	SECTION 4
%----------------------------------------------------------------------------------------
\section{Wyniki pomiarów}

Dla każdej każdej wartości napięcia (175,200,225,250 [V]) lampy regulowaliśmy natężenie prądu tak aby wiązka trafiała w poszczególne szczebelki oddalone o (2,3,4,5[mm]) od źródła. Pomiary wykonywaliśmy trzy razy, za każdym razem uzyskując dokładnie te same wartości.

\begin{table}[htbp]
\centering
\begin{tabular}{|c|c|c|c|c|}
\hline
\multicolumn{5}{|c|}{Wyniki pomiarów}  \\ \hline
U[V] & \multicolumn{4}{|c|}{175} \\ \hline
r[cm] & 0.02 & 0.03 & 0.04 & 0.05 \\ \hline
I[A] & 3.6 & 2.3 & 1.6 & 1.3 \\ \hline

U[V] & \multicolumn{4}{|c|}{200} \\ \hline
r[cm] & 0.02 & 0.03 & 0.04 & 0.05 \\ \hline
I[A] & 3.8 & 2.4 & 1.8 & 1.4 \\ \hline 

U[V] & \multicolumn{4}{|c|}{225} \\ \hline
r[cm] & 0.02 & 0.03 & 0.04 & 0.05 \\ \hline
I[A] & 4 & 2.6 & 1.9 & 1.5 \\ \hline

U[V] & \multicolumn{4}{|c|}{250} \\ \hline
r[cm] & 0.02 & 0.03 & 0.04 & 0.05 \\ \hline
I[A] & 4.1 & 2.7 & 2 & 1.6 \\ \hline
\end{tabular}
\caption{Wyniki pomiaru natężenia prądu w cewce dla poszczególnych napięć przyspieszających elektrony w wiązce i odległości punktów przecięcia z podziałką od lampy }
\label{}
\end{table}


%----------------------------------------------------------------------------------------
%	SECTION 5 - WYNIKI OPRACOWANIE
%----------------------------------------------------------------------------------------

\section{Opracowanie wyników}
\subsection{Obliczenie ładunku właściwego elektronu $\frac{e}{m}$ na podstawie wykonanych pomiarów}

Wiedząc, że na cewce mieliśmy 14 warstw drutu i w ramach każdej warstwy 11 zwojów otrzymujemy że na każdej cewce znajdowało się 154 zwojów.

Ładunek właściwy elektronu dla każdego pomiaru wyliczamy ze wzoru:
\begin{equation}
	\frac{e}{m} = 2.480 \cdot 10^{12} \cdot \frac{UR^2}{n^2I^2r^2}
\end{equation}
Gdzie n to liczba zwojów w cewce, U - napięcie przyspieszające, R - promień cewek, I - natężenie płynącego w cewkach prądu, r - promień tworzonego przez pole magnetyczne okręgu z elektronów.

\clearpage

Otrzymujemy wyniki zaprezentowane w tabeli:
\begin{table}[htbp]
\centering
\begin{tabular}{|c|r|r|r|r|}
\hline
U [V] & \multicolumn{4}{|c|}{$\frac{e}{m}$ [$\frac{C}{kg}$]} \\ \hline
175 & $1.41 \cdot 10^{11}$ & $1.54 \cdot 10^{11}$ & $1.79 \cdot 10^{11}$ & $1.73 \cdot 10^{11}$ \\ \hline
200 & $1.45 \cdot 10^{11}$ & $1.61 \cdot 10^{11}$ & $1.61 \cdot 10^{11}$ & $1.71 \cdot 10^{11}$ \\ \hline
225 & $1.47 \cdot 10^{11}$ & $1.55 \cdot 10^{11}$ & $1.63 \cdot 10^{11}$ & $1.67 \cdot 10^{11}$ \\ \hline
250 & $1.56 \cdot 10^{11}$ & $1.59 \cdot 10^{11}$ & $1.63 \cdot 10^{11}$ & $1.63 \cdot 10^{11}$ \\ \hline
\end{tabular}
\caption{Obliczone wartości ładunku właściwego elektronu}
\label{}
\end{table}

Znając masę elektronu można wyliczyć ładunek elektronu, bardziej elementarnej wartości.

\begin{table}[htbp]
\centering
\begin{tabular}{|c|r|r|r|r|}
\hline
U [V] & \multicolumn{4}{|c|}{e [C]} \\ \hline
175 & $1.29 \cdot 10^{-19}$ & $1.40 \cdot 10^{-19}$ & $1.63 \cdot 10^{-19}$ & $1.58 \cdot 10^{-19}$ \\ \hline
200 & $1.32 \cdot 10^{-19}$ &$1.47 \cdot 10^{-19}$ & $1.47 \cdot 10^{-19}$ & $1.56 \cdot 10^{-19}$ \\ \hline
225 & $1.34 \cdot 10^{-19}$ & $1.41 \cdot 10^{-19}$ & $1.48 \cdot 10^{-19}$ & $1.52 \cdot 10^{-19}$ \\ \hline
250 & $1.42 \cdot 10^{-19}$ & $1.45 \cdot 10^{-19}$ & $1.49 \cdot 10^{-19}$ & $1.49 \cdot 10^{-19}$ \\ \hline
\end{tabular}
\caption{Obliczone wartości ładunku elektronu przy skorzystaniu z tabelarycznej wartości masy elektronu m = 9.11 \cdot 10^{-31} [kg]}
\label{}
\end{table}

\subsection{Obliczenie średniej wartości ładunku właściwego i ładunku elektronu}

Wyliczamy średnią z obliczonych wartości:
\begin{itemize}
\item $\frac{e}{m} = -1.6 \cdot 10^{11} \frac{A \cdot s}{kg}$
\item $e = -1.37 \cdot 10^{-19} A \cdot s$
\end{itemize}

Pozostaje nam obliczyć średni błąd kwadratowy wyników, a ten wynosi: $1.02 \cdot 10^{10} \frac{C}{kg}$

Nasz wynik niewiele odbiega od tabelarycznej wartości wynoszącej $1.76 \cdot 10^{11}$. Już przy przyjęciu za k = 1.51 wynik eksperymentu zawiera się w przedziale błędu: $(-1.91 \cdot 10^{11} \frac{C}{kg}, -1.60 \cdot 10^{11} \frac{C}{kg})$

%----------------------------------------------------------------------------------------
%	SECTION 6
%----------------------------------------------------------------------------------------
\section{Wnioski}

Dzięki eksperymentowi otrzymaliśmy wynik -1.37 \cdot 10^{-19} [\frac{A \cdot s}{kg}] mieszczący się przy użyciu niepewności rozszerzonej w przedziale $(-1.91 \cdot 10^{11} \frac{C}{kg}, -1.60 \cdot 10^{11} \frac{C}{kg})$. 

Korzystając z tabelarycznej wartości masy spoczynkowej elektronu m = 9.11 \cdot 10^{-31} [kg] udało się także obliczyć e = $e = -1.37 \cdot 10^{-19} A \cdot s$.

Warto zaznaczyć, że duża podziałka wewnątrz lapmy oraz amperomierz dający wyniki o tylko dwóch miejscach znaczących mocno wpłynął na niepewność pomiaru. Doświadczenie okazało się jednak efektywną (oraz efektowną) metodą wyznaczania ładunku właściwego elektronu i da się ją porównać z metodą Thomsona.

%----------------------------------------------------------------------------------------
%	BIBLIOGRAPHY
%----------------------------------------------------------------------------------------

\bibliographystyle{apalike}

\bibliography{sample}
%----------------------------------------------------------------------------------------


\end{document}
